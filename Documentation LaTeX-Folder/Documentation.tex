\documentclass{article}
\usepackage[utf8]{inputenc}
\usepackage[ngerman]{babel}
\usepackage[hidelinks]{hyperref}
\usepackage{graphicx}
\usepackage{listings}
\usepackage{enumitem}
\usepackage{xcolor}
\usepackage{subcaption}
\usepackage{float}
\usepackage{url}
\usepackage{lastpage}
\usepackage{fancyhdr}
\usepackage{svg}
\usepackage{wrapfig}

\title{Dokumentation zum Werkstück A in Betriebssysteme und Rechnernetze : Schiffe-Versenken}
\author{Pascal Lupo, Gamachu Tufa, Jean-Gabriel Hanania}
\date{\today}

% Einstellung von Kopf- und Fußzeile
\pagestyle{fancy}
\fancyhf{}
\rhead{SS2021}
\lhead{Pascal Lupo, Gamachu Tufa, Jean-Gabriel Hanania}
\lfoot{Dokumentation zum Werkstück A von BS{\&}RN}
\rfoot{Seite \thepage\ von \pageref{LastPage}}
\renewcommand{\headrulewidth}{0.4pt}
\renewcommand{\footrulewidth}{0.4pt}

% Farben für die Darstellung von Quellcode
\definecolor{codedarkgrey}{rgb}{0.2,0.2,0.2}
\definecolor{codelightgrey}{rgb}{0.5,0.5,0.5}
\definecolor{backgroundcolour}{rgb}{0.95,0.95,0.95}

\lstdefinestyle{mystyle}{
    backgroundcolor=\color{backgroundcolour},
    commentstyle=\color{codelightgrey},
    keywordstyle=\color{blue},
    numberstyle=\tiny\color{black},
    stringstyle=\color{codedarkgrey},
    basicstyle=\ttfamily\footnotesize,
    breakatwhitespace=false,
    breaklines=true,
    captionpos=b,
    keepspaces=true,
    numbers=left,
    numbersep=5pt,
    showspaces=false,
    showstringspaces=false,
    showtabs=false,
    tabsize=2,
    aboveskip=15pt,
    belowskip=10pt
}

\lstset{style=mystyle}


% Paragrafeneinzuck entfernen
\setlength{\parindent}{0em}

\begin{document}

\maketitle
\thispagestyle{empty} % Damit erste Seite weder Fuß- noch Kopfzeile hat

% Zusammenfassung/Abstract
\begin{abstract}
    In dem vorliegenden Dokument wird beschrieben, wie das Spiel Schiffe-Versenken in Python programmiert wurde. Die größeren Herausforderungen dieses Projektes werden dargestellt. Es wird vor allem auf dem Aufbau und die Darstellung des Spieles, die Künstliche Intelligenz mit verschiedenen Schwierigkeitsstufen für das Spiel gegen dem Computer und das Einbauen einer Zeitbegrenzung für jeden Zug eingegangen.
\end{abstract}

% Einleitung
\section*{}
    Im Rahmen des Werkstücks A des Moduls Betriebssysteme und Rechnernetze SS2021 wurde das Spiel Schiffe-Versenken in Python programmiert. Schiffe-Versenken ist ein Spiel, dass üblicher Weise mit Stift und Papier gespielt wird, in dem jeder Spieler seine ''Schiffe'' auf einen verdecktem Spielbrett verteilen. Die Spieler ''schießen'' dann abwechselnd auf Felder des gegnerischen Spielbrettes, bis einer alle Schiffe des Gegners aufgedeckt und damit gewonnen hat .


\section{Aufbau des Spieles}
    Die erste Herausforderung dieses Projektes ist es, einen sinnvollen Aufbau des Spieles und eine Klare Darstellung zu finden, mit der jederzeit ersichtlich ist, was der Spielstand ist, welcher Spieler dran ist, wohin schon geschossen wurde und welche Möglichkeiten noch offen stehen.

\subsection{Speichern des Spielstandes}
    Zuerst ist es nötig alle wichtigen Informationen zu erkennen und eine Speichermethode zu wählen. Für das Spiel im Allgemein ist es sinnvoll, die Standardwerte der Einstellungen zu Speichern, z.B. die Größe der Spielfelder und die Anzahl an Schiffe die jeder Spieler besitzen soll. Dazu wird in Python ein sogenanntes Dictionary hergestellt. Das ermöglicht ein schnelles und übersichtliches Aufrufen der eingespeicherten Werte mit einem eindeutig genannten Schlüssel. In diesem Fall werden in \verb$settings_values['board_size']$ die Spielfeldgröße 10 und in \verb$settings_values['number_of_ships']$ die Anzahl 5 gespeichert. Das stellt  die üblichen Werte dieses Spieles dar. Diese lassen sich aber nach belieben in den Einstellungen ändern. Bei dem Start eines neuen Spieles wird dann basierend auf diese Einstellungen eine Liste aller möglichen Felder erzeugt (Listing \ref{lst:possible input}). Mit dieser lässt sich jede Feldeingabe der Spieler leicht abgleichen um Fehler abzufangen.

\begin{lstlisting}[language=Python, caption = Herstellung einer Liste allen möglichen Felder, label=lst:possible input]
 possible_input = []
 for y in range(settings_values['board_size']):
    for x in range(settings_values['board_size']):
        possible_input.append(chr(65 + y) + str(x + 1))
\end{lstlisting}

    Jeder Spieler bekommt auch ein Python-Dictionary. In diesen werden alle wichtige Informationen zum Spieler und seinen Spielstand gespeichert, d. h.:

\setlist{}%nolistsep} % Wenn wir denn Abstand entfernen wollen
\begin{itemize}
\itemsep0em
  \item Den vom Spieler ausgesuchtem Name unter dem Schlüssel \verb$'name'$
  \item Eine Tabelle, auf der die Schiffe platziert und die Schüsse des Gegners eingespeichert werden bei \verb$'board'$
  \item Ein Tabelle der versuchten Schüsse bei \verb$'guesses'$
  \item Die Anzahl an Schiffe die er und sein Gegner übrig haben bei \verb$'ships_left'$ und \verb$'enemy_ships_left'$
  \item Eine Liste der möglichen Felder, aus der nach und nach die Felder entfernt werden, auf denen der Spieler feuert bei \verb$'not_yet_tried'$
\end{itemize}

\subsection{Speicherung der Schiffpositionen}
    Am Anfang des Spieles wird eine List von Schiffen ausgesucht basierend auf die eingestellte Anzahl an Schiffen pro Spieler. Diese Listen sind vorgefertigt und jeder Eintrag darin ist ein Python-Dictionary, das ein ein bestimmtes Schiff Darstellen soll. Dort steht wie das Schiff heißt, mit welchem Kürzel es in der Tabelle repräsentiert wird und wie lang es ist.\\

\par
    Jeder Spieler entscheidet dann, ob er seine Schiffe selber platzieren möchte oder ob Diese zufällig verteilt werden sollen. Wählt er die erste Option wird er dazu aufgefordert ein Feld auszusuchen auf der die Spitze des Schiffes liegen soll. Das Programm berechnet dann ob an dieser Stelle genug Platz ist und gibt eine Liste möglicher Felder zurück, wo das andere Ende des Schiffes liegen kann. Nur wenn der Spieler für beides passende Werte eingibt, wird das Schiff mit dem passendem Kürzel in der Tabelle \verb$'board'$ des Spielers eingetragen. Die Zufallsverteilung erfolgt ähnlich. Jedes Wert wird nur zufällig aus denn Listen gültiger eingaben ausgesucht.

\subsection{Darstellung des Spieles}
    Jeder Spieler hat zwei Tabellen, in denen sein aktuellen Spielstand gespeichert ist.

\begin{figure}[H]
    \begin{subfigure}{0.50\textwidth}
    \centering
    \includesvg[width=0.40\textwidth]{./Anzeige/Player Board.svg}
    \caption{Spielfeld}
    \label{fig:board}
    \end{subfigure}
    \begin{subfigure}{0.50\textwidth}
    \centering
    \includesvg[width=0.40\textwidth]{./Anzeige/Player Guesses.svg}
    \caption{Schussversuche}
    \label{fig:guesses}
\end{subfigure}
\caption{Beispiel eines Spielstandes, das in Tabellen eines Spielers gespeichert ist}
\label{fig:Boards}
\end{figure}

\par
    In der ersten Tabelle (Abbildung \ref{fig:board}) stehen die Position und der Zustand der eigenen Schiffe. Diese werden mit einem einzigartigen Kürzel bezeichnet um sie von einander unterscheiden zu können (bspw. Su1 für Submarine 1 und Pa1 für Patrol Boat 1). An den Schiffteilen, die der Gegner schon getroffen hat, wird \verb$_Hit$ angefügt. In der zweiten Tabelle (Abbildung \ref{fig:guesses}) werden die Schussversuche gespeichert. Das Kürzel WG (Wrong Guess) makiert die Schüsse, die verfehlt haben und CG (Correct Guess) steht für die Felder wo ein gegnerisches Schiff getroffen wurde.\newline

\par
    Daraus wird dann ein Bauplan hergestellt. In diesem werden nicht nur die Daten aus den zwei Tabelle eingefügt, sondern auch die Abstände, die bei dem Ergebnis angezeigt werden sollen, die Beschriftung der Zeilen und Spalten und die Linie, die das Raster des Spielfeldes darstellen.

\begin{figure}[H]
    \centering
    \includesvg[width=.8\textwidth]{./Anzeige/Blueprint.svg}
    \caption{Bauplanergebnis aus dem Spielstand von Abbildung \ref{fig:Boards}}
    \label{fig:blueprint}
\end{figure}

\par
    Zum einfügen der verschiedenen Zeichen an der richtigen Stelle, wird für jede im Spielfeld liegenden (x,y) Koordinaten Combination mithilfe der Modulo 2 Funktion berechnet ob die Koordinaten gerade oder ungerade Zahlen sind. In dem nachfolgenden Listing \ref{lst:Grid example} wird bspw. für jedes Feld das im richtigem Bereich liegt und ein ungerade x-Koordinate besitzt eine Vertikale Linie eingefügt.

\begin{lstlisting}[language=Python, caption =Beispiel zum Einfügen von bestimmten Teilen des Spielfeldrasters, label=lst:Grid example]
for y in range(blueprint_height):
    for x in range(blueprint_width):
        # Filling in the vertical walls for both boards. They all have odd x coordinate
            if x % 2 == 1 and ((2 < x < (len(player['board']) * 2 + 4)) or (
                    (len(player['board']) * 2 + 4 + board_spacing) < x < blueprint_width)):
                blueprint[y][x] = "|"
\end{lstlisting}

\par
    Der Zwischenschritt des Bauplanes ermöglicht eine höher Kontrolle der Anzeige; z.B. liegen alle Schiffe durch ihre Länge über mindestens eine Linie des Rasters und würden dadurch in der Anzeige unterbrochen sein. Das würde für Unklarheiten sorgen sobald zwei Schiffe aneinander liegen würden. In dem Bauplan wird für jede Linie aus dem Spielfeld nachgeschaut, ob in den dazu angrenzenden Felder Teile eines gleichen Schiffes liegen. Damit kann dieses Zeichen gegebenenfalls dann mit ein einem passendem Schiffzeichen ersetzt werden. Da bspw. in der Abbildung \ref{fig:blueprint} die Felder A1 und B1 beide in der Spieler Tabelle mit ''Pa1''versehen sind, wird die Linie zwischen diesen Felder mit einem Schiffteil ersetzt.

\begin{figure}[H]
    \centering
    \includesvg[width=0.55\textwidth]{./Anzeige/Anzeigeergebnis.svg}
    \caption{Anzeigeergebnis}
    \label{fig:Result}
\end{figure}

    Die Abbildung \ref{fig:Result} ist das im Spiel angezeigt Ergebnis aus dem Spielstand der Abbildung \ref{fig:Boards}. Zum Darstellen der Schiffe und der Raster Linien kommen die sogannten ''Box-drawing character''\footnote{\url{https://en.wikipedia.org/wiki/Box-drawing_character}} des Unicode Standards im Einsatz. Bei der Ausgabe wird über Spielfelder angezeigt, welcher Spieler dran ist und wie viele Schiffe er und sein Gegner noch besitzen. All diese Werte lassen sich aus dem jeweiligen Python-Dictionary vom aktiven Spieler auslesen.


\section{PvE Game}
       Der PvE Modus ist dafür da, dass eine Person gegen einen computergesteuertem Spieler - einen sogenannte Bot - spielen kann. In diesem Modus kann man außerdem auswählen, wie gut die künstliche Intelligenz des Computer sein soll. Diese Einstellung soll Gegenspieler unterschiedlicher Level imitieren können, und damit das allgemeine Spielerlebnis abwechslungsreicher machen. Die KI des Bots ist in drei Schwierigkeitsgrade unterteilt: ''Einfach'', ''Mittel'' und ''Schwer''.

\subsection{Probleme bei der Umsetzung}
    Einer der größeren Probleme bei der Umsetzung des PvE Modus war es, eine auf dem Level des Spieler anpassbare künstliche Intelligenz zu integrieren. Wie sollte die Abstufen der verschiedenen Schwierigkeitslevel aussehen? Soll der Bot auf die umliegenden Felder schießen, wenn er ein Schiff getroffen hat und nicht einfach weiter zufällig auf das gegnerische Spielfeld schießen? Wie kann der Computer im schwerstem Modus taktisch vorgehen, um möglichst wenige Spielzüge zu tätigen um ein Schiff zu finden und versenken, wenn der Spieler den ''Schwer'' Modus einstellt?

\subsection{Umsetzung}
    Der Schwierigkeitsmodus ''Einfach'' soll einen unerfahrener Spieler imitieren der willkürlich auf das Feld feuert. Daher schießt der Bot komplett zufällig, ohne irgendwelche Kriterien oder anderes zu beachten. Dies erfolgt mit der \verb$random.choice$()-Funktion von Python, die aus der \verb$'not_yet_tried'$-Liste des Bots ein zufälliges Feld auswählt und es beschießt.\\

\par
    Der ''Mittel'' Schwierigkeitsmodus stellt einen Gelegenheits Spieler dar. Dafür schießt der Bot zuerst auch zufällig auf das Feld. Trifft er aber ein Schiff, wird geschaut welche anliegenden Felder noch nicht beschossen wurden und zu der \verb$next_shot$ Liste hinzugefügt. Daraus werden in den nächsten Zügen Felder zufällig gewählt. Alle Treffer speichert der Computer in seiner Liste \verb$'current_target'$. Falls es beim beschießen der umliegenden Felder wieder zu einem Treffer kommt, dann schaut das Programm, ob die beiden Treffer die gleiche y-Koordinate oder die gleiche x-Koordinate haben. In dem ersten Fall liegt das Schiff wahrscheinlich horizontal auf dem Spielfeld und im zweiten Fall wahrscheinlich vertikal. Nun schießt der Bot auf die anliegenden Felder mit der gleichen y- bzw. x-Koordinate. Wenn nun das Schiff versenkt wird, leert der Bot seine \verb$'current_target'$-Liste und fängt wieder an, zufällig zu schießen bis er einen neuen Treffer bekommt. Falls aber es aber keine freien anliegenden Felder mehr gibt - d. h., die \verb$next_shot$ Liste kommt leer zurück - aber es wurde kein Schiff versenkt, dann geht der Computer davon aus, dass es sich hier um mehrere aneinander liegende Schiffe handelt. Nun kopiert er die Felder aus \verb$'current_target'$ in einer neuen Liste namens \verb$'possible_target'$. Der Computer behandelt jedes dieser Felder als getrenntes Schiffes, und geht jedes davon durch bis jeweils ein Schiff versenkt wird. Danach geht es wieder mit zufälligem Schießen weiter.\\

\par
    Beim Schwierigkeitsmodus ''Schwer'' wird ein erfahrener Spieler simuliert, der effizient spielt und versucht in möglichst wenig Züge alle Schiffe des Gegners aufzudecken. Dafür wird erkannt das der Bot nur auf jedes zweite Feld schießen muss, da das kleinste Schiff zwei Felder lang ist. Falls er schon alle Zwei-Felder Schiffe versenkt hat, ändert sich seine Vorgehensweise und er schießt nur noch auf jedes dritte Feld. Das wird alles in der \verb$smart_random_shot()$-Funktion definiert. Diese prüft, was das kleinste Schiff des Gegners ist und baut darauf basierend verschiedene ''Grids''. Wenn das kleinste Schiff zwei Felder groß ist, sehen die zwei möglichen Grids so aus, wie die schwarzen und weißen Felder eines Schachbrettes. Der Computer zählt nach, wie viele offene Felder die jeweiligen Grids haben, und sucht sich die mit der kleinsten Anzahl aus. In dem er dann seine zufällige Schüsse nur auf diesem Raster betätigt, minimiert er die Anzahl an nötigen Versuchen um alle Schiffe zu finden. Wenn das kleinste übrige Schiffe drei Felder lang ist, ergeben sich die sechs mögliche Grids (Abbildung \ref{fig:3 field ship grids}). Diese bestehen aus diagonalen Linien, die zwei Kästchen Abstand voneinander haben.

\begin{figure}[H]
    \centering
    \includesvg[width=0.95\textwidth]{./Hunting Grids/All Hunting Grids for 3 field ships alt.svg}
    \caption{Mögliche Grids wenn das kleinste Schiff drei Felder lang ist}
    \label{fig:3 field ship grids}
\end{figure}

\begin{wrapfigure}{r}{0.50\textwidth}
    \centering
    \includesvg[width=0.85\linewidth]{./Hunting Grids/Grid Wechsel Beispiel.svg}
    \caption{Beispiel eines Grid-Wechsels}
    \label{fig:Grid}
\end{wrapfigure}

\par
    Es kann passieren, dass die ideale Grid sich beim Spielen ändert. Ein Beispiel dazu zeigt die Abbildung: \ref{fig:Grid}. Zuerst beschießt der Bot die weißen Kästchen, da es von diesen Feldern weniger gibt als von den schwarzen. So beginnt der Bot hier mit B1 und erzielt ein Treffer. Nun erkennt der Computer, dass dort ein Schiff seien muss und beschießt die umliegenden Felder. Der zweite Schuss geht auf B2 daneben. Mit den nächsten beiden Schüssen A1 und C1 versenkt er das Drei-Feld Schiff. Jetzt sucht der Bot nach dem letztem Schiff und wechselt dafür zur schwarzen Grid, denn jetzt sind nur noch zwei schwarze Kästchen und 3 weiße Kästchen übrig. Damit braucht er nur noch zwei Schüsse um sicher zu sein, das Schiff zu treffen.\\

\par
    Im Listing 3 ist ein Teil der \verb$smart_random_shot()$-Funktion zusehen, die bei dem ''Schwer'' Modus im Einsatz kommt wenn das kleinste Schiff noch am Lebem ist. Dort wird gezeigt, wie die zwei möglichen Grids berechnet und dann miteinander verglichen werden.
\setlist{}%nolistsep} % Wenn wir denn Abstand entfernen wollen
\begin{itemize}
\itemsep0em
    \item Zuerst werden zwei leer Listen herstellt, die die zwei möglichen Grids darstellen sollen. Diese werden \verb$hunting_grid00$ und \verb$hunting_grid01$ benannt.

    \item Nun wird für jedes leere Feld nachgeschaut, zu welchen Grid es gehört, und wird der entsprechenden Liste hinzugefügt. Dafür berechnet das Programm, ob die x-und-y Koordinaten durch Modulo 2 das gleiche Ergebnis haben, also entweder beide 0 oder beide 1 ergeben. Diese Felder werden alle in der \verb$hunting_grid_00$-Liste gespeichert. Die Anderen Felder, also die, bei denen die x-und-y Koordinaten nicht den selben Wert ergeben, wenn sie mit Modulo 2 verrechnet werden, speichert der Computer in der \verb$hunting_grid_01$-Liste.

    \item Im nächsten Schritt wählt er einer der zwei Listen zufällig als sein ''Haupt''-\verb$hunting_grid$ aus. Damit soll, im Falle dass die zwei Listen gleich lang sind, sichergestellt werden, dass nicht immer die gleiche benutzt wird.

    \item Diese zufällig ausgewählte List, wird dann mit beiden verglichen. Der Computer wechselt dann nur zur anderen Grid wenn diese weniger Felder in ihrer Liste stehen hat.

    \item Der Computer kann dann ein Feld zufällig aus seiner \verb$hunting_grid$-Liste ziehen um dorthin zu feuern
\end{itemize}

\begin{lstlisting}[language=Python, caption = Herstellung der zwei möglich Grids und deren Vergleich]
 hunting_grid_00 = []
 hunting_grid_01 = []

 # Checks every fields, and adds the empty one to the grid its part of
 for y in range(len(bot['guesses'])):
     for x in range(len(bot['guesses'])):
         if bot['guesses'][y][x] == "0":
             if (y % 2 == 0 and x % 2 == 0) or (y % 2 == 1 and x % 2 == 1):
                 hunting_grid_00.append(chr(65 + y) + str(x + 1))
             else:
                 hunting_grid_01.append(chr(65 + y) + str(x + 1))

 # Finally, it checks which list is the smallest of the 2, and sets it as our hunting grid
 # One Grid is chosen at random to reduce predicting possibilities
 hunting_grid = random.choice([hunting_grid_01, hunting_grid_00])
 if len(hunting_grid) > len(hunting_grid_01):
     hunting_grid = hunting_grid_01
 elif len(hunting_grid) > len(hunting_grid_00):
    hunting_grid = hunting_grid_00

 return random.choice(hunting_grid)
\end{lstlisting}

\section{PvP Game}
\par
    Beim PvP Modus, ist es möglich, dass 2 Personen an einem Gerät gegeneinander spielen können. In dem Modus können beide Spieler einen Namen auswählen und müssen abwechselnd am Computer ran. Dabei soll sichergestellt werden, dass die Spieler zu keinem Zeitpunkt das Spielfeld des Gegners zu sehen bekommen.
\subsection{Umsetzung}
    Dieses Spielmodus sieht in der Umsetzung dem PvE-Spiel sehr ähnlich. Damit die Spieler  die Position der Schiffe des anderen nicht nie erfahren können, wird zwischen jedem Spielzug der Bildschirm ''geleert'', und die Spieler werden dazu aufgefordert die Plätze zu tauschen bevor es weiter geht. Da ein Löschen der ganzen Konsole sehr umständlich wäre, wurde das stattdessen mit der \verb$clear_screen()$-Funktion gemacht. Diese gibt 50 leere Zeilen aus, um somit das Spielfeld des Gegners zumindestens aus dem direkt sichtbaren bereich Konsole zu entfernen.


\section{Counter}
    Die Funktion ist dafür da, um die Zeit der Benutzereingabe zeitlich zu begrenzen. Sobald der Spieler aufgefordert wir ein gegnerisches Feld zu beschießen,  läuft ein Countdown runter, der aufzeigt wie viel Zeit noch übrig bleibt. Der Zeitraum wurde auf 15 Sekunden festgelegt. Sollte der Spieler innerhalb der gegebenen Zeit ein gegnerisches Schiff beschießen, wird der Countdown beendet und das Spiel fortgesetzt, in dem der andere Spieler nun aufgefordert wird ein gegnerisches Feld zu beschießen. Sollte der Spieler bis Ablauf der Zeit noch kein Feld ausgewählt haben, wird dem Spieler angezeigt, dass seine Zeit abgelaufen ist und er nun die Enter-Taste drücken soll. Durch das Drücken der Enter-Taste wird ein zufälliges Feld beschossen und das Spiel fortgesetzt.
    Diese Funktion wird jedes Mal aufgerufen, wenn der Spieler ein Feld beschießen muss.

\subsection{Problemstellung}
    Ein Problem der Aufgabenstellung war, dass zwei Prozesse gleichzeitig laufen müssen. Zum einen muss der Countdown von 15 bis 0 runterlaufen, zum anderen muss gleichzeitig auf einen Benutzereingabe gewartet werden. Zu beginn gab es das Problem, dass der Countdown erst gestartet ist, nach dem es eine Benutzereingabe gab. Dies machte den Countdown jedoch unbrauchbar, da es parallel laufen muss.

\subsection{Lösungsversuch}
    Um nun mehrere Prozesse gleichzeitig laufen zu lassen musste eine andere Lösung her. Durch etwas Recherche wurde die Benutzung von ''threads'' als Möglichkeit gefunden. Diese sollten einen parallelen Ablauf von mehreren Prozessen möglich machen. Die Datei ist wie folgt aufgebaut:
    \setlist{}%nolistsep} % Wenn wir denn Abstand entfernen wollen
\begin{itemize}
\itemsep0em
    \item
    Es gibt die \verb$ask()$-Funktion. Diese Funktion nimmt die Benutzereingabe auf und gibt sie später, am Ende zurück.
    \item
    Die Funktion \verb$exit(msg)$ wird aufgerufen, wenn eine Benutzereingabe erfolgt ist oder wenn keine Eingabe erfolgt ist. In beiden Fällen gibt die Funktion einen Textwert zurück, der als Parameter gegeben werden muss. Es wird beispielsweise ausgegeben, dass die Zeit abgelaufen ist und nun ein zufälliges Feld beschossen wird oder es wird ausgegeben, auf welches Feld geschossen wurde.
    \item
    Die Funktion \verb$countdown()$ gibt die verbliebene Zeit an. Innerhalb der Funktion ist eine Endlosschleife die läuft, bis der Stop-Wert null erreicht wird. Der Wert beginnt bei 15 und wird  jede Runde, mit einer Verzögerung von 2 Sekunden um den Wert 2 verringert. Bei jeden Schleifendurchgang wird die aktuell verbliebene Zeit ausgegeben. Wichtig  ist hierbei zusagen, dass in der Gesamten zeit eine Benutzereingabe parallel möglich ist.
    \item
    Die Funktion \verb$close_if_time_pass(seconds)$ wird als letztes aufgerufen und gibt aus, dass die Zeit abgelaufen ist.
    \item
    Alle diese Funktionen werden in der Wichtigsten von allen, der \verb$main()$ Funktion, aufgerufen.
    \end{itemize}
    \par
    In der Funktion \verb$main()$, wurde, wie bereits erwähnt, mit ''threads'' gearbeitet. Es waren zwei ''threads'' nötig. Der Erste ist dafür da, um nach Ablauf der Zeit (in unserem Fall 15 Sekunden) auszugeben, dass die Zeit abgelaufen ist und nun vom Spiel selbst, ein zufälliges Feld ausgewählt wird. Der zweite ''thread'' ist dazu da, den Countdown zu starten. Sobald die ''threads'' gestartet werden, wird die Funktion \verb$ask()$ aufgerufen, welche auf die Benutzereingabe wartet. Der Rückgabewert aus der Funktion \verb$ask()$, wird gespeichert in der Variable \verb$user_input$. Es wird nach Ablauf der Zeit geprüft, ob die Variable keinen Wert hat. Sollte dies der Fall sein, wird ein zufälliges Feld beschossen, mit der Funktion \verb$random_ship_attack$, aus der Date gamefunctions. Sollte die Variable jedoch einen Wert haben, wird dieser Wert als Rückgabewert genutzt und somit das eingegeben Feld beschossen.


\begin{lstlisting}[language=Python, caption=Main Funktion des Countdowns]
def main(player, language):
    global start_sign, check

    # bool variables that are needed to start
    start_sign = True
    check = False

    # define close_if_time_pass as a threading function, 15 as an argument
    t = threading.Thread(target=close_if_time_pass, args=(15, player, language,))
    t2 = threading.Thread(target=countdown, args=(language,))

    # start threading
    t2.start()
    t.start()

    # ask for input
    user_input = ask()

    # if there was no user input, the player will attack a random field automatically
    if len(user_input) < 1:
        user_input = random.choice(player['not_yet_tried'])

    # bool variables that are needed to stop
    check = True
    start_sign = False

    return user_input
\end{lstlisting}

\section{Schlusswort}

\end{document}